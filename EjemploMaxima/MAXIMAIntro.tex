\documentclass{beamer}
\usepackage[ansinew]{inputenc} % Acepta caracteres en castellano
\usepackage[spanish]{babel}    % silabea palabras castellanas
%\usepackage[colorlinks=true,urlcolor=blue,linkcolor=blue]{hyperref} 
%\usepackage[hyperref]{beamerarticle}
% \usepackage{beamerthemesplit} // Activate for custom appearance

\title{Un paseo por MAXIMA. \\ Un Manipulador simb'olico del domino p'ublico}
\author{Hern'an Asorey y Luis A. N'u\~nez}
\date{Mayo 2013}

\begin{document}

\frame{\titlepage}

\section[Indice]{}
\frame{\tableofcontents}

\section{�qu'e son los manipuladores simb�licos?}
\frame
{
  \frametitle{�qu'e son los manipuladores simb�licos?}
   Son ambientes de trabajo cient�fico que permiten: 
   \begin{enumerate}
  \item realizar c'alculos matem�ticos  con polinomios y matrices. 
  \item resolver ecuaciones y sistemas de algebraicas,
  \item definir, derivar e integrar funciones.
  \item graficar funciones complejas 
\end{enumerate}
Como siempre existen paquetes comerciales, sofisticados como 
\begin{itemize}
  \item MAPLE \url{http://www.maplesoft.com} 
  \item Mathematica \url{http://www.wolfram.com} 
  \item 
\end{itemize}
}
\section{�que son los manipuladores simb�licos?}
\frame
{
  \frametitle{Features of the Beamer Class}

  \begin{itemize}
  \item<1-> Normal LaTeX class.
  \item<2-> Easy overlays.
  \item<3-> No external programs needed.      
  \end{itemize}
}
\end{document}
